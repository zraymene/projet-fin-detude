%\chapter*{Résumé} 
\newpage
\thispagestyle{empty}
\begin{center}
{\Large \textbf{Résumé}}
\end{center}
%
La détection d'objets est une technique importante dans les domaines de la vision par ordinateur et de l'apprentissage automatique, qui jouent un rôle important dans la technologie moderne. Le terme « détection d'objet » est utilisé de manière interchangeable avec d'autres techniques telles que la segmentation, mais dans notre cas, nous nous concentrerons sur la détection d'objet par boîte englobante. Cela a commencé avec des algorithmes simples qui recherchaient des caractéristiques définies manuellement, comme l'algorithme de Viola-Jones publié en 2001. Après le boom du deep learning, provoqué par l'augmentation drastique des ressources de calcul permettant de mettre en œuvre ces algorithmes complexes, de nombreuses méthodes ont été introduites, tous basés sur le réseau neuronal convolutif "CNN" et ils sont divisés en 2 types : les premiers sont appelés détecteurs "à deux étapes", où la première étape fait la proposition d'objet et la deuxième étape fait la classification. Ce type de détecteur a une grande précision mais est plus lent. Les détecteurs 'Two-Step' les plus connus sont la famille R-CNN. D'autre part, les détecteurs 'One-step' éliminent l'étape intermédiaire de proposition d'objet et effectuent directement la classification en considérant toutes les positions dans les images comme des objets potentiels. Ces détecteurs sont beaucoup plus rapides que leurs homologues mais ont une précision moindre. Les détecteurs 'One-Step' les plus célèbres sont la famille YOLO. L'objectif principal de cette thèse est de tester et de comparer les détecteurs 'One-Step' de la famille YOLO, qui sont YOLOv3, YOLOv4 et YOLOv5 dans la détection des URL. L'évaluation de ces modèles est basée sur les métriques de la courbe Précision x Rappel et de la précision moyenne (AP) qui décrivent les performances du modèle avec une plus grande précision.

\vspace{0.5cm}
\textbf{Mots clés}: Détection d'objets, Boîte englobante, Apprentissage profond, Réseaux de neurones convolutifs (CNN), Détecteurs en deux étapes, Famille R-CNN, Détecteurs en une étape, Famille YOLO, YOLOv3, YOLOv4, YOLOv5, Détection d'URL.