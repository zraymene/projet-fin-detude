\chapter*{Conclusion générale}
\pagestyle{fancy}
\fancyhead[L]{}
\fancyhead[R]{Conclusion générale}
\renewcommand{\headrulewidth}{1pt}
\fancyfoot[C]{\thepage}
\section*{Conclusion générale}
Dans ce travail, nous avons étudié un thème très abordé en détection d'objets et apprentissage profond qui s'applique  sur ensemble des images, 

Dans le cadre de ce mémoire, premièrement nous avons présenté la  définition  de la détection  d'objets  et ses  applications.

ensuite, nous avons introduit  les modèles de détection d'objets  (3 méthodes ) avant Période de détection par apprentissage profond (après 2014 ) ,dans cette période  la détection d'objets est regroupée en deux classes : la "détection en deux étapes (RCNN-SPP NET -FPN)" et la "détection en une étape (YOLO-SSD)".

Également, nous avons présenté  les Mesures d'évaluation des systèmes de détection d'objets. 

Enfin, nous décrivons la création de l'ensemble de données et les différentes étapes d'apprentissage à l'aide de YOLOv3, YOLOv4 et YOLOv5, ainsi que les résultats des tests.

Notre objectif était de mettre en place une architecture permettant reconnaissance d’url dans différents difficultés.

Les résultats expérimentaux exprimés en terme de plusieurs métriques dans le domaine   de la reconnaissance  d’url, nous a permis de déduire les conclusions suivantes:
\paragraph{-} Les 3 modèles ont donné des résultats très satisfaisants dans le test de la base de données, mais le meilleur est YOLOv4. 
\paragraph{-} Les 3 modèles ont donné $0.0\%$ (AP) dans le test de rotation à $90^\circ$.
\paragraph{-} Les 3 modèles ont donné $0,0\%$ (AP) au test du Manuscrit.

pour améliorer les résultats , nous proposons de : 

\paragraph{-} D’augmenter la taille de l'ensemble de données.
\paragraph{-}ajouter des images avec des difficultés à l'ensemble de données.
\paragraph{-} D’améliorer et d'ajouter le nombre d'itérations dans la phase d’apprentissage.
\paragraph{-} Utiliser les nouvelles versions de YOLO et comparer les résultats.


