\chapter*{Introduction générale}
\newpage
\pagestyle{fancy}
\fancyhead[L]{}
\fancyhead[R]{Introduction générale}
\renewcommand{\headrulewidth}{1pt}
\fancyfoot[C]{\thepage}


\section*{Introduction}
La détection d'objets comme étape principale dans n'importe quel processus  de reconnaissance, est la procédure consistant à déterminer l'instance de la classe à laquelle appartient l'objet et à estimer l'emplacement de l'objet en affichant le cadre de délimitation autour de l'objet. La détection d'une instance unique de classe à partir d'une image est appelée détection d'objet à classe unique, tandis que la détection des classes de tous les objets présents dans l'image est connue sous le nom de détection d'objet multi-classes. 

Récemment, la détection d'objets basée sur l'apprentissage profond a atteint de très bonnes performances. Cependant, il existe de nombreux défis avec les images prises dans le monde réel, tels que le bruit, l'occlusion partielle/complète, les conditions d'éclairage, la pose, etc. Ces problèmes ont un impact important sur la détection des objets et doivent être traités lors de la détection d'objet. 

L'objectif principal de ce mémoire est d'exploiter le domaine de l'apprentissage profond et son application à la détection d'objets. Puis, en utilisant des panneaux de signalisation comme exemple, le but principal de ce mémoire est d'implémenter et d'évaluer les performances du modèle YOLO pour divers situations, en appliquant des techniques de traitement d'images traditionnelles pour simuler les problèmes existant dans les images prises dans scènes réelles.

\section*{Structure du document}
Le mémoire est composé de trois chapitres et une conclusion:

\begin{itemize}
\item Le premier chapitre introduit le domaine de la détection d'objets.
\item Le deuxième chapitre est consacré à la détection d'objets en utilisant les outils de l'apprentissage profond. Il présente les principales techniques qui ont été développées dans ce contexte, ainsi que les principales bases d'images et métriques utilisées pour l'évaluation.
\item Le troisième chapitre décrit les différentes expérimentations, ainsi que les résultats obtenus et leur synthèse.
\item La conclusion conclut le travail réalisé et présente les différentes perspectives.
\end{itemize}