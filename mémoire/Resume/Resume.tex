%\chapter*{Résumé} 
\newpage
\thispagestyle{empty}
\begin{center}
{\Large \textbf{Résumé}}
\end{center}
%

La détection d'objets est une technique importante dans le domaine de la vision par ordinateur, car elle est considérée comme étape nécessaire dans n'importe quel processus de reconnaissance. C'est la procédure consistant à déterminer l'instance de la classe à laquelle appartient l'objet et à estimer son emplacement en affichant sa boîte englobante. 

Il est largement approuvé que les progrès de la détection d'objets ont généralement traversé deux périodes historiques : "la période de détection d'objets traditionnelle (avant 2014)", où la détection  a été effectuée grâce à des techniques classiques d'apprentissage automatique, et "la période de détection basée sur l'apprentissage profond (après 2014)", où les techniques classiques d'apprentissage automatique ont été complètement remplacées par des méthodes basées sur des réseaux de neurones profonds. 

Dans ce mémoire,nous nous concentrerons sur la détection d'objet basée sur l'apprentissage profond. L'objectif principal  est de réaliser une étude comparative de trois modèles de la famille YOLO,  déjà prouvés leur efficacité dans ce domaine de détection d'objets, qui sont YOLOv3, YOLOv4 et YOLOv5 dans le contexte de la détection des URL dans des photos prises par un téléphone mobile. 

Les résultats expérimentaux exprimés en terme de précision moyenne, ont montré la capacité de généralisation des trois   modèles YOLOv3, YOLOv4 et YOLOv5, en plus, la stabilité du modèle YOLOV4 vis à vis plusieurs difficultés rajoutées aux images.

\vspace{0.5cm}
\textbf{Mots clés}: Détection d'objets, Apprentissage profond, Réseaux de neurones convolutifs (CNN), YOLO, détection d'URL.