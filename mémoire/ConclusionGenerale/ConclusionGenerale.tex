\chapter*{Conclusion générale}
\newpage
\pagestyle{fancy}
\fancyhead[L]{}
\fancyhead[R]{Conclusion générale}
\renewcommand{\headrulewidth}{1pt}
\fancyfoot[C]{\thepage}
\section*{Conclusion générale}
Dans ce travail, nous avons étudié un thème très abordé en traitement d’images couleur qui est la segmentation sémantique d’images. L’intérêt de ce thème est de transformer l’image en informations interprétables.

Dans le cadre de ce mémoire, premièrement, nous avons présenté un aperçu sur la caractérisation de l’image couleur et ces éléments de géométrie discrète. Ces caractéristiques sont à la base de la majorité des traitements qu’on peut appliquer à une image couleur.

Ensuite, nous avons introduit les différentes méthodes classiques utilisées en segmentation d’images couleurs. Appliquées à une même image, ces méthodes ne donnent pas toujours le même résultat. Chaque méthode à ses propres caractéristiques et fonctionne mieux sur un type d’images que d’autre. 

Egalement, nous avons présenté l’application du concept de deep learning pour la segmentation sémantiques des images. 

Enfin, nous avons décrit les différentes expérimentations, ainsi que les résultats et leur synthèse. L'évaluation est réalisée sur une base de donnée CamVid destinée pour la compréhension des scènes de route/conduite, c'est une base d'image bien connue dans le domaine. 

Notre objectif était d'implémenter une architecture convolutive, encodeurs-décodeurs, basée sur le réseau neuronal convolutif VGG16 pour l'étiquetage au niveau des pixels, et de la tester avec différentes profondeurs, à savoir: 2,3, 4 et 5.

Les résultats expérimentaux exprimés en terme de plusieurs métriques reconnues dans le domaine de la segmentation sémantique des images, nous a permis de déduire les conclusions suivantes: 

\begin{itemize}
\item L'approche SegNet peut donner des résultats très stasfaisants.
\item Elle est directement liée à la profondeur du réseau.
\item Elle est influencée par le nombre de pixels dans la classe.
\end{itemize} 

Comme perspectives, nous proposons de :
\begin{enumerate}
\item Etendre cette étude à d'autre domaines, en utilisant d'autres bases de données.
\item Tester la méthode en termes de temps d'exécution.
\item Essayer de réduire l'influence du nombre de pixels dans la classe sur les performance de la méthode.
\end{enumerate}

