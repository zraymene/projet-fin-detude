%\chapter*{Abstract}
\newpage
\thispagestyle{empty}
\begin{center}
{\Large \textbf{Abstract}}
\end{center}
Object detection is an important technique in the fields of computer vision and machine learning, which play an important role in modern technology.
The term 'object detection' is used interchangeably with other techniques like segmentation, but in our case we will focus on bounding-box object detection.
It began with simple algorithms that looked for manually set characteristics, like the algorithm of Viola-Jones published in 2001. After the deep learning boom, caused by the drastic increase in computing resources that allowed to implement these complex algorithms, many methods were introduced, all of them based on Convolutional Neural Network "CNN" and they are divided into 2 types:
the first ones are called 'Two-step' detecteurs, where the first step does the object proposal and the second step does the classification. This type of detector has great precision but is slower. The most famous 'Two-Step' detectors are the R-CNN family. 
On the other hand, 'One-step' detecteurs eliminate the intermediate step of object proposal and do the classification directly by considering all positions in images as potential objects. These detecteurs are much faster than their counterparts but have lower precision. The most famous 'One-Step' detecteurs are the YOLO Family. 
The main aim of this dissertation is to test and compare 'One-Step' detecteurs in the YOLO family, which are YOLOv3, YOLOv4 and YOLOv5 in detecting URLs. 
The evaluation of these models is based on the metrics of Precison x Recall curve and average precision (AP) which describe the model's performance with higher precision. 

\vspace{1cm}
\textbf{Keywords}: Object Detection, Bounding-box, Deep learning, Convolutional Neural Networks (CNN), Two-Step detecteurs, R-CNN Family, One-Step detecteurs, YOLO family, YOLOv3, YOLOv4, YOLOv5, URL detection.


