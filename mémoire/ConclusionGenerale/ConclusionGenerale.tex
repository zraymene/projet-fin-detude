\chapter*{Conclusion générale}
\pagestyle{fancy}
\fancyhead[L]{}
\fancyhead[R]{Conclusion générale}
\renewcommand{\headrulewidth}{1pt}
\fancyfoot[C]{\thepage}
\section*{Conclusion générale}
Dans ce travail, nous avons étudié un thème très intéressant dans le domaine de la vision par ordinateur, à savoir  la détection d'objets, qui constitue une étape principale dans n'importe quel processus  de reconnaissance.\\

Dans le cadre de ce mémoire, premièrement, nous avons présenté une vue d'ensemble de la détection  d'objets, ses  applications, ainsi que les différentes méthodes classiques utilisées.

Ensuite, nous avons introduit  le méthodes récentes de la détection d'objets qui sont basées principalement sur l'apprentissage en profondeur, et qui sont regroupées en  deux classes : la détection en deux étapes  et la détection en une étape . En plus de certain nombre de mesures d'évaluation des systèmes de détection d'objets. 


Enfin, nous avons décrit les différentes expérimentations, ainsi que les résultats des différents tests et leur synthèse. 


Notre objectif était de tester la capacité de généralisation, ainsi que la stabilité de trois   modèles YOLOv3, YOLOv4 et YOLOv5, pour le contexte de la détection des URLs dans des photos prises par un téléphone mobile.

Les résultats expérimentaux exprimés en terme de précision moyenne, nous ont permis  de déduire les conclusions suivantes:
\paragraph{-} Les 3 modèles ont donné des résultats de généralisation très satisfaisants, et le meilleur est YOLOv4. 

Concernant la stabilité pou plusieurs difficultés: 

\paragraph{-} Les 3 modèles n'ont pas reconnu complètement les URLs  tournées par un angle de rotation à $90^\circ$, où la précision moyenne atteinte est: $0.0\%$.

\paragraph{-} Également pour les URL manuscrites, les 3 modèles ont fourni une précision moyenne de  $0,0\%$.

\paragraph{-}Enfin, pour améliorer ces résultats , nous proposons de : 
\begin{itemize}
\item Augmenter la taille de l'ensemble de données.
\item Ajouter des images avec des difficultés à l'ensemble de données d'entrainement.
\item Changer les paramètres d'entraînement.
\item Utiliser d'autres versions de la famille YOLO, même d'autres modèles de la famille des détecteurs en deux étapes.
\end{itemize}


