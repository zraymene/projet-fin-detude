%\chapter*{Abstract}
\newpage
\thispagestyle{empty}
\begin{center}
{\Large \textbf{Abstract}}
\end{center}

Object detection is an important technique in the field of computer vision, as it is considered a necessary step in any recognition process. It is the procedure of determining the instance of the class to which the object belongs and estimating its location by displaying its bounding box.

It is widely agreed that advances in object detection have generally gone through two historical periods: "the traditional object detection period (pre-2014)", where detection was performed through classical machine learning techniques, and "the deep learning-based detection period (after 2014)", where classical machine learning techniques have been completely replaced by methods based on deep neural networks.

In this thesis, we will focus on object detection based on deep learning. The main objective is to carry out a comparative study of three models of the YOLO family, already proven to be effective in this field of object detection that are YOLOv3, YOLOv4, and YOLOv5 in the context of the detection of URLs in photos taken by a mobile phone.

The experimental results, expressed in terms of average precision, showed the generalization capacity of the three models YOLOv3, YOLOv4, and YOLOv5, in addition,  the stability of the YOLOV4 model vis-a-vis to several difficulties added to the images.


\vspace{0.5cm}
\textbf{Keywords}: Object Detection, Deep Learning, Convolutional Neural Networks (CNN), YOLO, URL Detection.
